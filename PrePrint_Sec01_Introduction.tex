\newpage


%//------ Section 01 -------------------------------------------------------------------------------------------------
\section{Introduction}
\label{sec:Section01-Intro}
%//-----------------------------------------------------------------------//

Colliding heavy nuclei at extremely high energy offers the possibility to form a very hot, dense and strongly interacting state of matter, called Quark Gluon Plasma (QGP). This exists for about $10^{-23}$ s \ref{bjorken_highly_1983}, which is far shorter than the readout time of the fastest detector -- the observation of such a state is therefore based on its remnants, meaning the collision products. QGP existence is associated to several signatures ; most of them have also been observed in high multiplicity proton-proton (pp) collisions, where no QGP is foreseen \ref{cms_collaboration_evidence_2015}\ref{alice_collaboration_enhanced_2017}\ref{alice_collaboration_multiplicity_2016}. This questions thorougly our representation of this state of nuclear matter, and calls for a review of the concepts : either the AA collisions/QGP physics picture must be further rooted on pp collisions, or pp QCD physics must be enriched with AA considerations. One way or the other, a better description of the continuum between pp and AA collisions is needed. Nowadays, several phenomenological models are following this approach, but more experimental input are needed (Ref needed). More precisely, this input takes the shape of multi-differential measurements of identified hadrons as a function of the rapidity, the transverse momentum and the multiplicity in charged particles. This type of analysis relies on excellent particle identification capabilites, which are provided by the ALICE detector.

One of the signature usually associated to QGP is the increase of the production of strange quarks, also called strangeness enhancement. Since strangeness is very sensitive to collectivity and the enhancement of the latter is the steepest for pp collisions, this signature might be the way to shed light on the collision dynamic in small systems. Thus, a focus will be given on measurement of the correlated production of strange hadrons in pp collisions. 

An example of a such measurement is proposed by Christian Bierlich : thanks to the recent developments in \Pythiaeight -- namely the color rope and color shoving mechanisms -- it is predicted that i) the \rmOmegaM [$sss$] abundancy increases when an \rmPhiMes [$s \bar{s}$] is present within the event, and ii) this increase is more prominent as the gap in rapidity between the two particles reduces. 

In this analysis, we propose to study the correlated production of strangeness via the measurement of the yield ratio of (\rmOmegas) and \rmPhiMes in proton-proton collisions at a center-of-mass energy $\sqrt{s} = 13 $ TeV from the LHC run II, using the ALICE detector. A comparison with various Monte-Carlo (MC) generators comes to complement the results presented in this paper. For the sake of brievty, ( \rmOmegas) will be denoted \rmOmegaPM, unless otherwise stated. This paper is organized as follows. In section \ref{sec:Section02}, the dataset used in this analysis is presented, followed by the event selections in section \ref{sec:Section03}. Further details about the analysis are showed in section \ref{sec:Section04} with the track and topological selections, and in section \ref{sec:Section05}, systematic study is exposed. Finally, the results are given in section \ref{sec:Section06}, accompanied by the comparison with phenomenological models and the discussion in section \ref{sec:Section07}.



