\documentclass[ALICE,manyauthors]{cernphprep}



%_____________________________________________________________________________________
%Pour les documentations sur les differents packages, voir le TeX catalogue on line 
%        http://texcatalogue.sarovar.org/index.html,
%_____________________________________________________________________________________


\usepackage[T1]{fontenc}    
    % for output font rendering, T1 rendering
    % textsc into section title http://en.wikibooks.org/wiki/LaTeX/Fonts#Font_encoding
% \usepackage{lmodern}
    % choose specific font, lmodern (default : if installed = cm-super, rendering ok!)
    % There is nothing to change in your document to use CM Super fonts (assuming they are installed), 
    %      they will get loaded automatically if you use T1 encoding. 
    %      For lmodern, you will need to load the package after the T1 encoding has been set
    
%\usepackage[latin1]{inputenc}
\usepackage[utf8]{inputenc} % for input source code, accented characters like French spelling...

\usepackage{graphicx,subfigure}
%\usepackage{subcaption} %for subfigures, added 10.dec.20

%\usepackage[english,francais]{babel}
\usepackage{amsmath}
\usepackage{amssymb}  % boldsymbol, special characters
\usepackage{mathrsfs} % pour le L du lagrangien ...
\usepackage{MnSymbol} % pour fivedots
\usepackage{bbding}   % pour checkmark spéciaux dans tableau d'inventaire
\usepackage{dictsym}  % pour dsaagricultural du tableau d'inventaire
\usepackage{manfnt}   % pour le signe "virage dangereux" \textlhdbend
%\usepackage{marvosym}  % Special Fonts, if need be. http://texdoc.net/texmf-dist/doc/fonts/marvosym/marvodoc.pdf
% \usepackage{eurosym}  % symbole de l'euro, \officialeuro
%     \DeclareUnicodeCharacter{20AC}{\euro}  % i.e. make the translation € UTF-8 char to be the LaTeX € symbol
\usepackage[retainorgcmds]{IEEEtrantools} % equation à la IEEE
\usepackage{multirow}
\usepackage{xspace} 
\usepackage{lineno}     % for line numbering
\usepackage[showframe=false]{geometry}
\usepackage{changepage} % for managing locally the width allowed for the text = useful for shifting leftwards too wide tables or figure
                        % e.g. \begin{adjustwidth}{-1cm}{} ... \end{adjustwidth}
\usepackage{threeparttable} % for enabling footnote within a table
\usepackage{pdflscape}  % Rotation of tables (better handling), for pdflatex
\usepackage{dcolumn}  % to align column on the decimal point
\usepackage{enumitem} % to be able to have enumerate a,b... A,B... i,ii...
                        % \begin{enumerate}[label=(\alph*)], [label=(\Alph*)], [label=(\roman*)]
\usepackage[normalem]{ulem} % pour tuner le soulignage, [normalem] = pour préserver le comportement normale de emph
\usepackage{longtable}


% after babel
\usepackage{datetime} % pour afficher l'heure avec la commande \currenttime = \xxivtime

%%%%%%%%%%%%%package Romain
% \usepackage[lofdepth,lotdepth,caption=false]{subfig}  
    % FIXME pb incompatible with hyperref apparently, enable to compile the document locally
    % https://tex.stackexchange.com/questions/129791/using-subfloat-with-hyperref
%%%%%%%%%%%%%end package Romain


% Package nécessaire pour les remerciements
% \usepackage{endnotes}
% \renewcommand{\notesname}{Notes de remerciements}



%%% OPTION - pdflatex compiler
\usepackage[pdftex,usenames,dvipsnames]{xcolor}
%\usepackage[pdftex]{graphicx}
\usepackage{epstopdf}
\DeclareGraphicsExtensions{.jpg,.eps,.png,.pdf}


%%% OPTION - Latex compiler
% \usepackage[usenames,dvipsnames]{color}
% \usepackage[dvips]{graphicx}
% \DeclareGraphicsExtensions{.jpg,.eps,.pdf,.png}
% %\DeclareGraphicsRule{.jpg}{eps}{.jpg.bb}{`./jpeg2ps/jpeg2ps -h #1}
% % % Truc de Julien : permet d'inclure des jpg au lieu d'eps ("wrapper" des jpeg en PS Level 2)
% % % Particulièrement utile pour les fichiers images volumineux 
% % % Voir également le shell script Jpg-DoBdngBox.sh et les fichiers *.jpg.bb
% 
% % \DeclareGraphicsRule{.eps.zip}{eps}{.eps.bb}{`unzip -p #1}%   zipped EPS
% % \DeclareGraphicsRule{.eps.gz}{eps}{.eps.bb}{`gunzip -c #1}%   gzipped EPS
% %         \DeclareGraphicsRule{.jpg}{eps}{.jpg.bb}{`convert #1 eps:-}%         JPEG
% %         \DeclareGraphicsRule{.gif}{eps}{gif.bb}{`convert #1 eps:-}%      GIF
% \DeclareGraphicsRule{.png}{eps}{.png.bb}{`convert #1 eps:-}%      PNG
% %         \DeclareGraphicsRule{.tif}{eps}{.bb}{`convert #1 eps:-}%      TIFF
% \DeclareGraphicsRule{.pdf}{eps}{.pdf.bb}{`convert #1 eps:-}%      PDF-graphics



\usepackage[bookmarks]{hyperref}
        % NOTE :
        % For colour choices : https://en.wikibooks.org/wiki/LaTeX/Colors

        \makeatletter
        \Hy@AtBeginDocument{%
        \def\@pdfborder{0 0 1}% Overrides border definition set with colorlinks=true
        \def\@pdfborderstyle{/S/S/W 1}% Overrides border style set with colorlinks=true
                                        % Hyperlink border style will be framed of width 1pt
                                        % https://tex.stackexchange.com/questions/26071/how-can-i-have-colored-and-underlined-links-with-hyperref
        }
        \makeatother

        \hypersetup{colorlinks=true, 
        % NOTE:
        %   - to have coloured frames + have coloured links, uncomment l.89+90 above and colorlinks=true
        %   - to have coloured frames + kill coloured links,   comment l.89+90 above and colorlinks=false
        %   - to kill coloured frames + have coloured links,   comment l.89+90 above and colorlinks=true
        % See https://tex.stackexchange.com/questions/50747/options-for-appearance-of-links-in-hyperref
                    linktocpage,
                    citebordercolor=ForestGreen,
                    linkbordercolor=Red,
                    urlbordercolor=Cerulean,
                    % menubordercolor= [rgb 1 0 0]
                    % filebordercolor= [rgb 0 .5 .5]
                    % runbordercolor= [rgb 0 .7 .7]
                    %allbordercolors=Red
                    citecolor=MidnightBlue, 
                    filecolor=MidnightBlue, 
                    linkcolor=MidnightBlue, 
                    urlcolor=MidnightBlue}
        % \hypersetup{colorlinks,  linktocpage, citecolor=Gray, filecolor=Gray, linkcolor=Gray, urlcolor=Gray} %FIXME : N&B
        \urlstyle{sf} % change the url font for sans serif sf, else rm
        
        

\usepackage[comma,square,numbers,sort&compress]{natbib}
% \usepackage[numbers, sort&compress]{mynatbib} %pour avoir une génération automatique de réf. bib [1-6] au lieu de lister [1, 2, 3, 4, 5, 6].
% \usepackage{hypernat} 
% L'usage de mynatbib = version locale modifiée de natbib est fait pour 
% empêcher la réinterprétation de la commande \newblock de la bibliographie
% En conséquence, tout ce qui est \newblock dans natbib.sty = changer pour "\\"

%\usepackage{array} 
        %pour avoir \extrarowheight = gestion de la hauteur de ligne (voir chap. 4)
        % Ajout de hauteur suppl aux lignes pour éviter que la ligne horizontale ne touche le texte.
        %\setlength{\extrarowheight}{3 pt}  à mettre au dessus du tableau considérée. 0 pt = valeur par défaut
        %Option m{largeur colonne} permet de centrer le texte verticalement (package array)
        
\usepackage{textpos}  % pour la realisation de la page de titre
        \setlength{\TPHorizModule}{10mm}
        \setlength{\TPVertModule}{10mm}
                        % definition de l'unite de base de longueur pour le placement avec textpos

\usepackage{epigraph} %allow rajout d'epigraphe (voir conclusion pour exemple)
        \setlength{\epigraphwidth}{80mm}
        \renewcommand{\epigraphsize}{\footnotesize}
        
%\usepackage{fancyhdr} %modification of page headers and footers
% Already loaded by cernphprep.cls

\pagestyle{fancy}
    \addtolength{\headheight}{\baselineskip}
    \renewcommand{\sectionmark}[1]{\markboth{\thesection.~~#1}{}}
    \renewcommand{\subsectionmark}[1]{\markright{\thesubsection.~~#1}}

\fancypagestyle{MainStyle}{%

    \fancyhf{}
    \fancyfoot[C]{\thepage}
    % \fancyhead[RE]{\small\leftmark}
    % \fancyhead[LO]{\small\rightmark}
%     \fancyhead[L]{\truncate{1.0\headwidth}{\small\rightmark }}
    \fancyhead[L]{\small\leftmark\\\footnotesize\rightmark}
    \fancyhead[R]{\truncate{1.0\headwidth}{\hyperlink{LinkToTOC}{to TOC} ~/~~\thepage}}
    \renewcommand{\headrulewidth}{0.2pt}
}




%\usepackage{tocbibind} %allow integration de biblio+index dans la TOC(a comparer avec addcontentsline)
%\usepackage{bibunits} %permet de faire une biblio par partie, chapitre, section...
        

%__________ Redefine style of the header top line, with section names
        
\def\MakeUppercase#1{{ \textsf{\small #1} }}
% MakeUppercase is already defined into LaTeX





%__________Mise en place des \newcommand generales
%\newcommand{\Bluecite}[1]{\textcolor{Blue}{\cite{#1}}}
\newcommand{\BoldSubSection}[1]{\noindent \textbf{\textsl{#1}}\\   \addcontentsline{toc}{subsection}{ \textsl{\textcolor{Gray}{\small #1 }}}  }

\newcommand{\SlantedSubSubSection}[1]{ -- \textsl{#1}\\   \addcontentsline{toc}{subsubsection}{ \textsl{\textcolor{Gray}{\small #1 }}}  }


\newcommand{\urlscpt}[1]{\hbox{\scriptsize \url{#1}}}
        % reduce the size of Internet address

\newcommand{\refmark}[1]{\hbox{\scriptsize $^{\ref{#1}}$}}
        %pour pouvoir faire une référence multiple a une note de bas de page
        % voir exemple dans le chap. 4

\renewcommand\descriptionlabel[1]{\hspace\labelsep\normalfont\itshape #1 :}
%in the environment "description", produce labels in italic, with colon at the end





%\renewcommand{\newblock}{\\}
% pour revenir à la ligne après chaque bloc, dans la bibliographie = réinterpréter newblock en \\

\newenvironment{BulletList}%
{ \begin{list}%
        {$\bullet$}%
        {\setlength{\labelwidth}{30pt}%
         \setlength{\leftmargin}{35pt}%
         \setlength{\itemsep}{\parsep}%
         \setlength{\topsep}{\parsep}}}
{ \end{list} }





%__________Definition of colours
% http://cloford.com/resources/colours/500col.htm

% in gray shades
\definecolor{DarkGray}{RGB}{60,60,60}
\definecolor{LightGray}{RGB}{145,145,145}

% in red shades
\definecolor{Sepia}{RGB}{94,38,18}
\definecolor{IndianRed}{RGB}{176,23,31}
\definecolor{OrangeRed4}{RGB}{139,37,0}
\definecolor{DarkRed}{RGB}{139,0,0}

% in orange shades
\definecolor{Orange2}{RGB}{238,154,0}
\definecolor{Goldenrod1}{RGB}{255,193,37}
\definecolor{Goldenrod2}{RGB}{238,180,34} 

% in blue shades
\definecolor{DarkSlateBlue}{RGB}{72,61,139}
\definecolor{Cobalt}{RGB}{61,89,171}
\definecolor{RoyalBlue4}{RGB}{39,64,139}
\definecolor{DodgerBlue4}{RGB}{16,78,139}
\definecolor{SteelBlue4}{RGB}{54,100,139}
\definecolor{DeepSkyBlue4}{RGB}{0,104,139}

% in green shades
\definecolor{LightGreen}{RGB}{0,200,0}





%__________Insertion of source codes


\usepackage{listings}
% In order to include source code from various prog language
% For documentation : 
%   http://en.wikibooks.org/wiki/LaTeX/Source_Code_Listings
%   http://www.ctan.org/tex-archive/macros/latex/contrib/listings/


\lstset{ %
  backgroundcolor=\color{white},          % choose the background color; you must add \usepackage{color} or \usepackage{xcolor}
  basicstyle=\footnotesize\ttfamily,      % the font and size that are used for the code
  breakatwhitespace=false,                % sets if automatic breaks should only happen at whitespace
  breaklines=true,                        % sets automatic line breaking
  captionpos=t,                           % sets the caption-position to top
  commentstyle=\color{LightGray}\upshape, % comment style
  deletekeywords={...},                   % if you want to delete keywords from the given language
  %escapeinside={\%*}{*)},                % if you want to add LaTeX within your code
  extendedchars=true,                     % lets you use non-ASCII characters; for 8-bits encodings only, does not work with UTF-8
  frame=tlbr,                             % adds a frame around the code : single, t b l r, T B L R
  keepspaces=false,                        % keeps spaces in text, useful for keeping indentation of code (possibly needs columns=flexible)
  keywordstyle=\bfseries\color{black},    % keyword style
  identifierstyle=,
  language=C,                        % the default language of the code
  %morekeywords={*,...},                  % if you want to add more keywords to the set
  numbers=left,                           % where to put the line-numbers; possible values are (none, left, right)
  numbersep=8pt,                          % how far the line-numbers are from the code
  numberstyle=\tiny\color{LightGray},     % the style that is used for the line-numbers
  rulecolor=\color{LightGray},            % if not set, the frame-color may be changed on line-breaks within not-black text (e.g. comments (green here))
  showspaces=false,                       % show spaces everywhere adding particular underscores; it overrides 'showstringspaces'
  showstringspaces=false,                 % underline spaces within strings only
  showtabs=false,                         % show tabs within strings adding particular underscores
  stepnumber=1,                           % the step between two line-numbers. If it's 1, each line will be numbered
  stringstyle=\color{Goldenrod2},         % string literal style
  tabsize=2,                              % sets default tabsize to 2 spaces
  caption=\lstname,                       % show the filename of files included with \lstinputlisting; also try caption instead of title
  xleftmargin=10pt,                       % size of the left margin
  belowcaptionskip=1.2\baselineskip,
  aboveskip=1\baselineskip,               % vertical skip above the listings envt
  belowskip=1\baselineskip,
}

\lstdefinestyle{customC++}{
  language=C++
}

\lstdefinestyle{customBash}{
  language=bash
}

\lstdefinestyle{customCmnd}{
  language=sh,
  identifierstyle=\color{blue},
  morecomment=[l][\color{LightGray}]{!\ } % define the rest of the whole line as comments
}

% look-up table to have the listings package fully compatible with UTF-8 extended char (extendedchars has to be true).
\lstset{literate=
  {á}{{\'a}}1 {é}{{\'e}}1 {í}{{\'i}}1 {ó}{{\'o}}1 {ú}{{\'u}}1
  {Á}{{\'A}}1 {É}{{\'E}}1 {Í}{{\'I}}1 {Ó}{{\'O}}1 {Ú}{{\'U}}1
  {à}{{\`a}}1 {è}{{\'e}}1 {ì}{{\`i}}1 {ò}{{\`o}}1 {ò}{{\`u}}1
  {À}{{\`A}}1 {È}{{\'E}}1 {Ì}{{\`I}}1 {Ò}{{\`O}}1 {Ò}{{\`U}}1
  {ä}{{\"a}}1 {ë}{{\"e}}1 {ï}{{\"i}}1 {ö}{{\"o}}1 {ü}{{\"u}}1
  {Ä}{{\"A}}1 {Ë}{{\"E}}1 {Ï}{{\"I}}1 {Ö}{{\"O}}1 {Ü}{{\"U}}1
  {â}{{\^a}}1 {ê}{{\^e}}1 {î}{{\^i}}1 {ô}{{\^o}}1 {û}{{\^u}}1
  {Â}{{\^A}}1 {Ê}{{\^E}}1 {Î}{{\^I}}1 {Ô}{{\^O}}1 {Û}{{\^U}}1
  {œ}{{\oe}}1 {Œ}{{\OE}}1 {æ}{{\ae}}1 {Æ}{{\AE}}1 {ß}{{\ss}}1
  {ç}{{\c c}}1 {Ç}{{\c C}}1 {ø}{{\o}}1 {å}{{\r a}}1 {Å}{{\r A}}1
  {€}{{\EUR}}1 {£}{{\pounds}}1 
}






%__________Mise en place de la structure en chap., section ... + toc

\renewcommand{\thepart} {\Alph{part}.}
%\renewcommand{\thechapter} {\Roman{chapter}}
\renewcommand{\thesection} {\Roman{section}}
\renewcommand{\thesubsection}   {~~\thesection-{\small \Alph{subsection}}}
\renewcommand{\thesubsubsection} {~~~~\underline{\thesection-{\small \Alph{subsection}}.{\footnotesize \roman{subsubsection}}}}
% pour avoir une structure du type "I.A -1.i" au lieu de "1.1.1.1"


% TOC normal
\setcounter{tocdepth}{3}     % Arret au niveau des subsubsection
\setcounter{secnumdepth}{3}  % Arret de la numerotation au niveau des subsubsections


% - 1.
\usepackage{titlesec}
% definition utilisateur du style des chap., sections, ...

        \usepackage{etoolbox}
        % Bug in sectionning : plus de numéros apparent dans le texte
        % Fix = http://tex.stackexchange.com/questions/299969/titlesec-loss-of-section-numbering-with-the-new-update-2016-03-15
        \makeatletter
        \patchcmd{\ttlh@hang}{\parindent\z@}{\parindent\z@\leavevmode}{}{}
        \patchcmd{\ttlh@hang}{\noindent}{}{}{}
        \makeatother


%font style = \sffamily, \ttfamily, \rmfamily
%font series = \bfseries, \mdseries
%font shape = \upshape, \itshape, \scshape, \slshape
% http://www.math.jussieu.fr/~goutet/latex/seance_6/seance_6.pdf

%\titleclass{\part}{straight}
\titleformat{\part}[display]
  {\normalfont\sffamily\huge\bfseries\color{OrangeRed4}}
  % FIXME {\normalfont\sffamily\huge\bfseries\color{Black}}
  {-- \partname\ \thepart}{20pt}{\Huge}

% \titleformat{\chapter}[display]
%   {\normalfont\sffamily\huge\bfseries\color{Sepia}}
%   % FIXME {\normalfont\sffamily\huge\bfseries\color{Black}}
%   {-- \chaptertitlename\ \thechapter~-- }{20pt}{\Huge}
  
%\titleclass{\section}{straight}  
\titleformat{\section}[hang]
  {\normalfont\rmfamily\Large\bfseries\color{OrangeRed4}}
  % FIXME {\normalfont\rmfamily\Large\bfseries\color{Black}}
  {\thesection}{1em}{}

\titleformat{\subsection}[hang]
  {\normalfont\rmfamily\large\bfseries\color{DarkGray}}
  % FIXME {\normalfont\rmfamily\large\bfseries\color{Black}}
  {\thesubsection}{1em}{}

\titleformat{\subsubsection}[hang]
  {\itshape\rmfamily\large\bfseries\color{LightGray}}
  % FIXME {\itshape\rmfamily\large\bfseries\color{Black}}
  {\thesubsubsection}{1em}{}

% - 2.
\usepackage{titletoc}
% definition utilisateur du style du TOC ...

\titlecontents{part}%
[2em]% retrait à gauche
{\addvspace{5em plus 0pt}\flushright\bfseries\color{MidnightBlue}}% matériel avant commun aux entrées numérotées ou pas
{\contentslabel{2.0em}}% avant lorsqu'il y a un numéro
{\hspace{-2.0em}}% avant lorsqu'il n'y a pas de numéro
{}% points de suspension et numéro de page
[\addvspace{2em}]% matériel après



\titlecontents{chapter}%
[2.5em]% retrait à gauche
{\addvspace{3em plus 0pt}\bfseries}% matériel avant commun aux entrées numérotées ou pas
{\contentslabel{2.5em}}% avant lorsqu'il y a un numéro
{\hspace{-2.5em}}% avant lorsqu'il n'y a pas de numéro
{\dotfill\contentspage}% points de suspension et numéro de page
[\addvspace{0pt}]% matériel après


\titlecontents{section}%
[4.5em]% retrait à gauche
{\addvspace{8pt}\mdseries\color{OrangeRed4}}% matériel avant commun aux entrées numérotées ou pas
{\contentslabel{3.5em}}% avant lorsqu'il y a un numéro
{\hspace{-3.5em}}% avant lorsqu'il n'y a pas de numéro
{\dotfill\contentspage}% points de suspension et numéro de page
[\addvspace{0pt}]% matériel après


\titlecontents{subsection}%
[5.5em]% retrait à gauche
{\mdseries\color{DarkGray}}% matériel avant commun aux entrées numérotées ou pas
{\contentslabel{4.5em}}% avant lorsqu'il y a un numéro
{\hspace{-4.5em}}% avant lorsqu'il n'y a pas de numéro
{\dotfill\contentspage}% points de suspension et numéro de page
[\addvspace{-3pt}]% matériel après


\titlecontents{subsubsection}%
[6.5em]% retrait à gauche
{\slshape\mdseries\color{LightGray}}% matériel avant commun aux entrées numérotées ou pas
{\contentslabel{5.5em}}% avant lorsqu'il y a un numéro
{\hspace{-5.5em}}% avant lorsqu'il n'y a pas de numéro
{\dotfill\contentspage}% points de suspension et numéro de page
[\addvspace{-3pt}]% matériel après


% - 3 : Corrections nécessaires pour les TOC partiels
\makeatletter
\AtBeginDocument{%
    \def\ttl@gobblecontents#1#2#3#4{\ignorespaces}%
}
\makeatother

% - 4 : Correction nécessaire pour améliorer la référence à une (sous-sous-)section

%\newcommand{\refSection}[1]{\mbox{\kern-0.1em \ref{#1}}}
\newcommand{\refSubSection}[1]{\mbox{\kern-0.6em \ref{#1}}}
\newcommand{\refSubSubSection}[1]{\mbox{\kern-1.2em \ref{#1}}}
        % suite à la redéfinition du style de la hiérarchie (II.A.1.i)
        % faire référence à un paragraphe laisse beaucoup d'espace devant la référence : "Dans le paragaphe      I.C.2.i"













%__________Option de draft : Définition de la version


\newcommand{\version}[2]{ [Version {#1} - {\scriptsize (git rev.{#2})} -  \today, \currenttime] }


% _____ Option 1 - Simple
% to get a light mark in the diagonal of every page
% drawback :    for people commenting on the pdf, the diagonal can mess up 
%               the selection of words you would like to comment on.
%               Said to be inconvenient.
% \usepackage{draftwatermark}
% \SetWatermarkLightness{0.90}
% \SetWatermarkAngle{90}
% \SetWatermarkScale{0.3}
% \SetWatermarkText{\today, \currenttime}


% _____ Option 2 - More complex : to get the draftwatermark in the right margin = https://ctan.org/pkg/background
\usepackage[color=gray]{background}
\backgroundsetup{
    position={+8.7cm,-6cm},
    firstpage=true, % does not work here apparently
    angle=-90,
    opacity=0.5,
    scale=2,
   %contents=Draft % the exact text is set in the master document
}







%__________Option de draft : notes en marge

\usepackage[colorinlistoftodos]{todonotes} % disable, obeyDraft, obeyFinal




\pagestyle{headings}





%_____________________________________________________________________________________
%Pour les documentations sur les differents packages, voir le TeX catalogue on line 
%        http://texcatalogue.sarovar.org/index.html,
%_____________________________________________________________________________________


%__________Definition of scientific commands

%
% 1 - some text editions
%
\newcommand {\stat}     {({\it stat.})~}
\newcommand {\syst}     {({\it syst.})~}

\newcommand{\ie}        {$i.e.$~}
\newcommand{\eg}        {$e.g.$~}
\newcommand{\Eg}        {$E.g.$~}
\newcommand{\fig}       {\textsc{f}ig.~}
\newcommand{\tab}       {\textsc{t}ab.~}
\newcommand{\chap}      {\textsc{c}hap.~}
\newcommand{\parag}     {\textsc{p}ar.~}
\newcommand{\eq}        {\textsc{e}q.~}
\newcommand{\opt}        {\textsc{o}pt.~}

\newcommand{\tune}      {\emph{tune}}
\newcommand{\tunes}     {\emph{tunes}\xspace}


\newcommand{\orderOf}[1]{\ensuremath{\mathcal{O}}(#1)}

\newcommand{\CheckGr}   {\textcolor{Green}{\normalsize \CheckmarkBold}} 
\newcommand{\SurpriseGr}{\textcolor{Green}{\textbf{!}{\scriptsize !}}}
\newcommand{\Wtf}       {\textcolor{IndianRed}{\textbf{?}!}}
\newcommand{\NB}        {\textcolor{IndianRed}{\HandRight}}
\newcommand{\Caution}   {\textcolor{IndianRed}{\scriptsize \textlhdbend}}
\newcommand{\NoWay}     {\textcolor{IndianRed}{\normalsize \XSolidBrush}}
\newcommand{\UnderWork} {\textcolor{Blue}{\Large \dsagricultural}}
\newcommand{\ToDo}      {\textcolor{Gray}{\scriptsize \textsc{ToDo}}}


%
% 2 - some notations
%
% \text{} is more general than \mahtrm{} : 
% it does not switch font to roman but just use the given font and write it straight (especially needed into title, sections..).
% http://tex.stackexchange.com/questions/98406/which-command-should-i-use-for-textual-subscripts-in-math-mode

% 2.1 - Physics quantities

\newcommand {\pT}           {\ensuremath{p_{\text{\textsc{t}}}}\xspace}
% \DeclareRobustCommand {\pT}        {\ensuremath{p_{\text{\textsc{t}}}}}
\newcommand {\pZ}           {\ensuremath{p_{\text{\textsc{z}}}}}
\newcommand {\pTlw}         {\ensuremath{p_{\text{\textsc{t}}}^{lw} }}
\newcommand {\pTIdx}[1]     {\ensuremath{p_{\text{\textsc{t},#1}}}}
\newcommand {\pTExp}[1]     {\ensuremath{p_{\text{\textsc{t}}}^{#1}}}
\newcommand {\pTIdxExp}[2]  {\ensuremath{p_{\text{\textsc{t},#1}}^{#2}}}
\newcommand {\pTof}[1]      {\ensuremath{p_{\text{\textsc{t}}} \text{(#1)}}}
\newcommand {\pTchJet}      {\ensuremath{p_{\text{\textsc{t},jet}}^{\text{ch}}}}
\newcommand {\pTchEmJet}    {\ensuremath{p_{\text{\textsc{t},jet}}^{\text{ch+em}}}}
\newcommand {\sigmapT}      {\mbox{$\sigma_{\pT}$}}
\newcommand {\meanpT}       {\ensuremath{\langle p_{\textsc{t}} \kern-0.1em\rangle}}
\newcommand {\mean}[1]      {\ensuremath{\langle #1 \kern-0.1em\rangle}} 
\newcommand {\sqrtSnn}      {\ensuremath{\sqrt{s_\text{\textsc{nn}}}}\xspace}
\newcommand {\sqrtS}        {\ensuremath{\sqrt{s}}\xspace}
\newcommand {\vTwo}         {\ensuremath{v_{\text{2}}}}
\newcommand {\vThree}       {\ensuremath{v_{\text{3}}}}
\newcommand {\vFour}        {\ensuremath{v_{\text{4}}}}
\newcommand {\vFive}        {\ensuremath{v_{\text{5}}}}
\newcommand {\vSix}         {\ensuremath{v_{\text{6}}}}
\newcommand {\vN}           {\ensuremath{v_{\text{n}}}}
\newcommand {\eT}           {\ensuremath{E_{\text{\textsc{t}}}}}
\newcommand {\mT}           {\ensuremath{m_{\text{\textsc{t}}}}}
\newcommand {\mTmZero}      {\ensuremath{m_{\text{\textsc{t}}} - m_0}}
\newcommand {\mInvIdx}[1]   {\mbox{$m_{ #1 }$}}
\newcommand {\mPart}[1]     {\mbox{$m[ #1 ]$}}
\newcommand {\sigmaM}[1]    {\mbox{$\sigma_m[ #1 ]$}}
\newcommand {\DeltaM}[1]    {\mbox{$\Delta m[ #1 ]$}}
\newcommand {\sigmaIdx}[1]  {\ensuremath{\sigma_{ #1 }}}

\newcommand {\Nsigma}[1]    {\ensuremath{n.\sigma_{ #1 }}}

\newcommand {\rap}          {\mbox{$y$}}
\newcommand {\rapLab}       {\mbox{$y_{\text{lab}}$}}
\newcommand {\rapCms}       {\mbox{$y_{\text{\textsc{cms} }}$}}
\newcommand {\absrap}       {\mbox{$\left | y \right | $}\xspace}
\newcommand {\rapPart}[1]   {\mbox{$\left | y\text{(#1)} \right | $}}
\newcommand {\rapXi}        {\mbox{$\left | y(\rmXi) \right | $}}
\newcommand {\rapJpsi}      {\mbox{$y_{\tiny \rmJpsi}$}}
\newcommand {\abspseudorap} {\mbox{$\left | \eta \right | $}\xspace}
\newcommand {\pseudorap}    {\mbox{$\eta$}\xspace}
\newcommand {\pseudorapLab} {\mbox{$\eta_{\,\text{lab}}$}}
\newcommand {\pseudorapCms} {\mbox{$\eta_{\text{\textsc{cms} }}$}}
\newcommand {\cTau}         {\ensuremath{c.\tau}\xspace}
\newcommand {\sigee}        {$\sigma_E$/$E$}

\newcommand {\oneOverpipT}  {\ensuremath{1/2\pi\pT}}
\newcommand {\crossSec}[1]  {\mbox{$\sigma_{\scriptsize \rm #1}$}}
\newcommand {\visCrossSec}[1]  {\mbox{$\sigma_{\scriptsize \rm #1}^{visible}$}}
\newcommand {\dsigmady}     {\ensuremath{\text{d}\sigma/\text{d}y}}
\newcommand {\dsigmadpt}    {\ensuremath{\text{d}^{2}\sigma/\text{d}\pT}}
\newcommand {\dsigmadptdy}  {\ensuremath{\text{d}^{2}\sigma/\text{d}\pT\text{d}y}}
\newcommand {\dsigmadptdeta}{\ensuremath{\text{d}^{2}\sigma/\text{d}\pT\text{d}\eta}}
\newcommand {\dsigmaXdy}[1] {\ensuremath{\text{d}\sigma\text{(#1)}/\text{d}y}}
\newcommand {\dsigmaXdpt}[1]{\ensuremath{\text{d}\sigma\text{(#1)}/\text{d}\pT}}
\newcommand {\dsigmaXdptdy}[1]  {\ensuremath{\text{d}^{2}\sigma\text{(#1)}/\text{d}\pT\text{d}y}}
\newcommand {\dNdy}         {\ensuremath{\text{d}N/\text{d}y}}
\newcommand {\dNJpsidy}     {\ensuremath{\text{d}N_{\textsc{\rmJpsi}}/\text{d}y}}
\newcommand {\dNdpt}        {\ensuremath{\text{d}N/\text{d}\pT }}
\newcommand {\dNXdptdy}[1]  {\ensuremath{\text{d}^{2}N\text{(#1)}/\text{d}\pT\text{d}y}}
\newcommand {\dNdptdy}      {\ensuremath{\text{d}^{2}N/\text{d}\pT\text{d}y }}
\newcommand {\dNdptdeta}    {\ensuremath{\text{d}^{2}N/\text{d}\pT\text{d}\eta }}
\newcommand {\fracdsigmadptdy}  {\ensuremath{ \frac{\text{d}^{2}\sigma}{\text{d}\pT\text{d}y}}}
\newcommand {\fracdNdptdy}  {\ensuremath{ \frac{\text{d}^{2}N}{\text{d}\pT\text{d}y } }}
\newcommand {\fracdNdy}     {\ensuremath{ \frac{\dN}{\dy}}}
\newcommand {\fracdNdyBold} {\ensuremath{ \frac{\bm{\dN}}{\bm{\dy}}}}

\newcommand {\dNdmtdy}      {\ensuremath{\text{d}^{2}N/\text{d}\mT\text{d}y }}
\newcommand {\dN}           {\ensuremath{\text{d}N }}
\newcommand {\Npp}          {\ensuremath{N_{\textsc{\pp}}}}
\newcommand {\dNsquared}    {\ensuremath{\text{d}^{2}N }}
\newcommand {\dsquared}     {\ensuremath{\text{d}^{2} }}
\newcommand {\dpT}          {\ensuremath{\text{d}\pT }}
\newcommand {\dy}           {\ensuremath{\text{d}y}}
\newcommand {\dNdyBold}     {\ensuremath{\bm{\dN/\dy}}}
\newcommand {\dNchdy}       {\ensuremath{\text{d}N_\text{ch}/\text{d}y }}
\newcommand {\dNchdeta}     {\ensuremath{\text{d}N_\text{ch}/\text{d}\eta }}
\newcommand {\dNchdptdeta}  {\ensuremath{\text{d}^{2}N_\text{ch}/\text{d}\pT\text{d}\eta }}
\newcommand {\RAA}          {\ensuremath{R_\text{AA}}}
\newcommand {\RpA}          {\ensuremath{R_\text{pA}}}
\newcommand {\RpPb}         {\ensuremath{R_\text{pPb}}}
\newcommand {\RPbp}         {\ensuremath{R_\text{Pbp}}}
\newcommand {\RAuAu}        {\ensuremath{R_\text{AuAu}}}
\newcommand {\RPbPb}        {\ensuremath{R_\text{PbPb}}}
\newcommand {\Rcp}          {\ensuremath{R_\text{CP}}}
\newcommand {\hPMVzsCorrel} {\ensuremath{(\text{\hPM-V0})}}
\newcommand {\hVzsCorrel}   {\ensuremath{(\text{h-V0})}}
\newcommand {\mPDG}[1]      {\ensuremath{m_{\textsc{pdg}}(#1)}}


\newcommand {\Ntracklet}    {\ensuremath{N_{\text{tracklets}}^{|\eta| < 1}}}
\newcommand {\Nevt}         {\ensuremath{N_\text{evt}}}
\newcommand {\NevtINEL}     {\ensuremath{N_\text{evt}(\textsc{inel})}}
\newcommand {\NevtNSD}      {\ensuremath{N_\text{evt}(\textsc{nsd})}}
\newcommand {\INEL}         {\ensuremath{\textsc{inel}}}
\newcommand {\INELZero}     {\ensuremath{\textsc{inel}>0}}
\newcommand {\NSD}          {\ensuremath{\textsc{nsd}}}
\newcommand {\dEdx}         {\ensuremath{\textup{d}E/\textup{d}x }}

\newcommand {\bsTe}      {\ensuremath{\bm{T_e}}}
\newcommand {\bsTb}      {\ensuremath{\bm{T_b}}}
\newcommand {\bsTt}      {\ensuremath{\bm{T_T}}}
\newcommand {\bsCt}      {\ensuremath{\bm{C_T}}}
\newcommand {\bsNt}      {\ensuremath{\bm{n}}}
\newcommand {\bsQt}      {\ensuremath{\bm{q}}}
\newcommand {\bsVt}      {\ensuremath{\bm{V}}}
\newcommand {\bsgVt}     {\ensuremath{\bm{g.V}}}
\newcommand {\bsNb}      {\ensuremath{\bm{n}}}
\newcommand {\bsFp}      {\ensuremath{\bm{f_P}}}
\newcommand {\bsCp}      {\ensuremath{\bm{C_P}}}

\newcommand {\Lint}         {\ensuremath{L_{\text{int}}}}

\newcommand {\rphi}         {\mbox{\ensuremath{(r,\varphi)}}}
\newcommand {\alphaS}       {\ensuremath{ \alpha_s}}
\newcommand {\chLeptonAsymm}{\ensuremath{ A_{\ell\ell} }}

\newcommand {\MeanNpart}    {\mbox{\ensuremath{\langle\kern-0.05em N_{part} \kern-0.05em \rangle}}}
\newcommand {\MeanNcoll}    {\mbox{\ensuremath{\langle\kern-0.05em N_{coll} \kern-0.05em \rangle}}}
\newcommand {\sigmaBarlow}  {\ensuremath{\sigma_{Barlow}}}
\newcommand {\sigmaStat}    {\ensuremath{\sigma_{stat}}}
\newcommand {\sigmaSyst}    {\ensuremath{\sigma_{syst}}}
\newcommand {\sigmaTot}     {\ensuremath{\sqrt{\sigmaStat^2 + \sigmaSyst^2}}}
\newcommand {\sigmaINELZero}{\ensuremath{\sigma/\sigma_{\text{INEL>0}}}\xspace}





% 2.2 - some generator Names

\newcommand{\Sherpa}        {\textsc{Sherpa}\xspace}
\newcommand{\Herwig}        {\textsc{Herwig++}\xspace}
\newcommand{\Epos}          {\textsc{Epos}\xspace}
\newcommand{\Pythia}        {\textsc{Pythia}\xspace}
\newcommand{\Pythiaeight}   {\Pythia~8\xspace}
\newcommand{\Rivet}         {\textsc{Rivet}\xspace}
\newcommand{\HepMC}         {\textsc{HepMc}\xspace}

% 2.3 - some generator tunes

\newcommand{\Monash}        {\textsc{Monash~2013}\xspace}
\newcommand{\Perugia}       {\textsc{Perugia~2011}\xspace}

% 2.4 - some propagator Names

\newcommand{\Geantfour}     {\textsc{Geant~4}\xspace}
\newcommand{\Geantthree}    {\textsc{Geant~3}\xspace}
\newcommand{\Fluka}         {\textsc{Fluka}\xspace}

%
% 3 - Collisions Systems
%
\newcommand {\pp}        {\ensuremath{\mbox{\text {p\kern-0.05em p}}}\xspace}
%\newcommand {\ppBoldMath} {\mbox{$\mathrm{ \mathbf p\kern-0.05em \mathbf p }$}}
\newcommand {\ppbar}     {\mbox{$\text{p}\overline{\text{p}}$}} 
\newcommand {\PbPb}      {\ensuremath{\mbox{\text{Pb--Pb}} }}
\newcommand {\AuAu}      {\ensuremath{\mbox{\text{Au--Au}} }}
\newcommand {\ArAr}      {\ensuremath{\mbox{\text{Ar--Ar}} }}
\newcommand {\CuCu}      {\ensuremath{\mbox{\text{Cu--Cu}} }}
\newcommand {\UU}        {\ensuremath{\mbox{\text{U--U}}   }}
\renewcommand {\AA}      {\ensuremath{\mbox{\text{A--A}}   }}
\newcommand {\pA}        {\ensuremath{\mbox{\text{p--A}}   }}
\newcommand {\dA}        {\ensuremath{\mbox{\text{d--A}}   }}
\newcommand {\pPb}       {\ensuremath{\mbox{\text{p--Pb}}  }}
\newcommand {\Pbp}       {\ensuremath{\mbox{\text{Pb--p}}  }}
\newcommand {\dAu}       {\ensuremath{\mbox{\text{d--Au}}  }}
\newcommand {\EplusEminus}      {\ee}
\newcommand {\ee}               {\mbox{$\text{e}^+\text{e}^-$}}
\newcommand {\MuPlusMuMinus}    {\mbox{$\mu^+\mu^-$}}


%
% 4 - some units
%
\newcommand {\massStyle}[1] {\mbox{\ensuremath{\text{#1}\kern-0.1em /\kern-0.12em c^2}}}
\newcommand {\mass}     {\massStyle{MeV}}
\newcommand {\kmass}    {\massStyle{keV}}
\newcommand {\mmass}    {\massStyle{MeV}}
\newcommand {\gmass}    {\massStyle{GeV}}

\newcommand {\unitStyle}[1] {\mbox{\ensuremath{\text{#1}}}}
\newcommand {\tev}      {\unitStyle{TeV}}
\newcommand {\gev}      {\unitStyle{GeV}}
\newcommand {\mev}      {\unitStyle{MeV}}
\newcommand {\kev}      {\unitStyle{keV}}
%\newcommand {\tevBoldMath}  {\mbox{${\rm \mathbf{TeV}}$}}
%\newcommand {\gevBoldMath}  {\mbox{${\rm \mathbf{GeV}}$}}

\newcommand {\momStyle}[1] {\mbox{\ensuremath{\text{#1}\kern-0.1em /\kern-0.12em c}}}
\newcommand {\mmom}     {\momStyle{MeV}}
\newcommand {\gmom}     {\momStyle{GeV}}

\newcommand {\fsec}       {\unitStyle{fs}}
\newcommand {\psec}       {\unitStyle{ps}}
\newcommand {\nsec}       {\unitStyle{ns}}
\newcommand {\musec}      {\mbox{$\mu\unitStyle{s}$}}
\newcommand {\millisec}   {\unitStyle{ms}}
\newcommand {\second}     {\unitStyle{s}}

\newcommand {\fmC}      {\mbox{$\unitStyle{fm}/\kern-0.12em c$}}

\newcommand {\fm}       {\unitStyle{fm}}
\newcommand {\nm}       {\unitStyle{nm}}
\newcommand {\mum}      {\mbox{$\mu\unitStyle{m}$}}
\newcommand {\mm}       {\unitStyle{mm}}
\newcommand {\cm}       {\unitStyle{cm}\xspace}
\newcommand {\m}        {\unitStyle{m}}

\newcommand {\cmq}      {\mbox{$\cm^2$}}
\newcommand {\mmq}      {\mbox{$\mm^2$}}
\newcommand {\mumq}     {\mbox{$\mum^2$}}

\newcommand {\fmCube}   {\mbox{\unitStyle{fm}$^3$}}

\newcommand {\mug}      {\mbox{$\mu\unitStyle{g}$}}
\newcommand {\mg}       {\unitStyle{mg}}
\newcommand {\gram}     {\unitStyle{g}}
\newcommand {\kg}       {\unitStyle{kg}}

\newcommand {\dens}     {\mbox{$\unitStyle{g}/\unitStyle{cm}^{3}$}}

\newcommand {\dg}       {\mbox{$\kern+0.1em ^\circ$}}


\newcommand {\lumi}     {\mbox{$\cm^{-2}\second^{-1}$}}

\newcommand {\barn}     {\unitStyle{b}}
\newcommand {\fb}       {\unitStyle{fb}}
\newcommand {\pb}       {\unitStyle{pb}}
\newcommand {\nb}       {\unitStyle{nb}}
\newcommand {\mub}      {\mbox{$\mu\unitStyle{b}$}}
\newcommand {\mb}       {\unitStyle{mb}}
\newcommand {\kb}       {\unitStyle{kb}}

\newcommand {\invmub}   {\mbox{$\mub^{-1}$}}
\newcommand {\invnb}    {\mbox{$\nb^{-1}$}}
\newcommand {\invpb}    {\mbox{$\pb^{-1}$}}
\newcommand {\invfb}    {\mbox{$\fb^{-1}$}}




%
% 5 - some particles
%
% For particles, there one may always use romant font -> mathrm

\newcommand{\hPM}           {\ensuremath{h^{\pm}}}
\newcommand{\ePlusMinus}    {\mbox{$\mathrm {e^{\pm}}$}}
\newcommand{\muPlusMinus}   {\mbox{$\mathrm {\mu^{\pm}}$}}

\newcommand{\piZero}        {\mbox{$\mathrm {\pi^0}$}}
\newcommand{\piMinus}       {\mbox{$\mathrm {\pi^-}$}}
\newcommand{\piPlus}        {\mbox{$\mathrm {\pi^+}$}}
\newcommand{\piPlusMinus}   {\mbox{$\mathrm {\pi^{\pm}}$}}
\newcommand{\rmPiPlusMinus} {\piPlusMinus}
\newcommand{\rmPiPM}        {\piPlusMinus}

\newcommand{\Kzs}        {\mbox{$\mathrm {K^0_S}$}}
\newcommand{\rmKzero}    {\Kzs}
\newcommand{\Kzl}        {\mbox{$\mathrm {K^0_L}$}}
\newcommand{\rmKzeroL}   {\Kzl}
\newcommand{\Kminus}     {\mbox{$\mathrm {K^-}$}}
\newcommand{\rmKminus}   {\Kminus}
\newcommand{\Kplus}      {\mbox{$\mathrm {K^+}$}}
\newcommand{\rmKplus}    {\Kplus}
\newcommand{\Kplusmin}   {\mbox{$\mathrm {K^{\pm}}$}}
\newcommand{\rmKpm}      {\Kplusmin}
\newcommand{\rmKstar}    {\mbox{$\mathrm{K}^*\mathrm{(892)}^0$}}
\newcommand{\rmPhiMes}   {\mbox{$\mathrm {\phi(1020)}$}\xspace}

\newcommand{\proton}    {\mbox{$\mathrm {p}$}}
\newcommand{\pbar}      {\mbox{$\mathrm {\overline{p}}$}}
% \newcommand{\pOrPbar}   {\mbox{$\mathrm{\vcenter{\offinterlineskip \vskip-0.1ex\hbox{\tiny \kern-0.05em- \kern-0.3em-} \vskip+0.2ex\hbox{p}}^{\protect \tiny ^{\hdots} \kern-0.75em _{+} \kern-.7em}}$ }}
\newcommand{\pOrPbar}   {\mbox{$\mathrm {p^{\pm}}$}}


\newcommand{\rmLambdaZ}         {\mbox{$\mathrm {\Lambda}$}}
\newcommand{\rmAlambdaZ}        {\mbox{$\mathrm {\overline{\Lambda}}$}}
\newcommand{\rmLambda}          {\mbox{$\mathrm {\Lambda}$}\xspace}
\newcommand{\rmAlambda}         {\mbox{$\mathrm {\overline{\Lambda}}$}\xspace}
\newcommand{\rmLambdas}         {\mbox{$\mathrm {\Lambda \kern-0.2em + \kern-0.2em \overline{\Lambda}}$}}
\newcommand{\ratioLamOverKzs}   {\rmLambda/\rmKzero}

\newcommand{\rmSigma}       {\mbox{$\mathrm {\Sigma}$}}
\newcommand{\rmSigmaM}      {\mbox{$\mathrm{\Sigma}^{-}$}}
\newcommand{\rmSigmaP}      {\mbox{$\mathrm{\Sigma}^{+}$}}
\newcommand{\rmSigmaZero}   {\mbox{$\mathrm{\Sigma}^{0}$}}
\newcommand{\rmSigmaMres}   {\mbox{$\mathrm{\Sigma(1385)}^{-}$}}
\newcommand{\rmSigmaPres}   {\mbox{$\mathrm{\Sigma(1385)}^{+}$}}


\newcommand{\rmXi}      {\mbox{$\mathrm{\Xi}$}\xspace}
\newcommand{\rmXiM}     {\mbox{$\mathrm{\Xi}^{-}$}\xspace}
\newcommand{\rmXiPM}    {\mbox{$\mathrm {\Xi^{\pm}}$}\xspace}
% \newcommand{\rmXiPM} { \mbox{$\kern-0.1em \mathrm{\vcenter{\offinterlineskip \vskip-1.0ex\hbox{\tiny \kern-0.05em- \kern-0.3em- \kern-0.3em-} \vskip+0.01ex\hbox{$\Xi$}}^{\protect \underline{\fivedots}}} \kern-0.1em$} }
% \newcommand{\rmXiPM} { \mbox{$\kern-0.0em \mathrm{\vcenter{\offinterlineskip \vskip-1.0ex\hbox{\tiny \kern+0.05em- \kern-0.3em- \kern-0.3em-} \vskip+0.1ex\hbox{$\Xi$}}^{\protect \tiny ^\fivedots \kern-0.75em _{\relbar} \kern-.3em}}$} }


% \newcommand{\rmXiPM} {\mbox{$\kern-0.1em \mathrm{ \protect \overset{ {\tiny \kern-0.09em- \kern-0.3em- \kern-0.3em-} }{\Xi}^{\protect \tiny ^\fivedots \kern-0.3em / \kern-0.2em _- \kern-0.25em}}$}}



% 
% \tiny ^+ \kern-0.3em / \kern-0.2em _- \kern-0.25em
\newcommand{\rmAxiP}    {\mbox{$\mathrm {\overline{\Xi}^{+}}$}}
\newcommand{\rmXis}     {\mbox{$\mathrm {\Xi^{-} \kern-0.3em + \kern-0.1em \overline{\Xi}^{+}}$}}
\newcommand{\rmXiZero}  {\mbox{$\mathrm {\Xi^{0}}$}}
\newcommand{\rmXiZ}     {\rmXiZero}
\newcommand{\rmXiZres}  {\mbox{$\mathrm {\Xi (1530)^{0}}$}}
\newcommand{\rmAxiZres} {\mbox{$\mathrm {\overline{\Xi} (1530)^{0}}$}}
\newcommand{\rmXiMres}  {\mbox{$\mathrm {\Xi (1530)^{-}}$}}
\newcommand{\rmAxiPres} {\mbox{$\mathrm {\overline{\Xi}(1530)^{+}}$}}

\newcommand{\rmOmega}   {\mbox{$\mathrm {\Omega}$}\xspace}
\newcommand{\rmOmegaM}  {\mbox{$\mathrm {\Omega^{-}}$}\xspace}
\newcommand{\rmAomegaP} {\mbox{$\mathrm {\overline{\Omega}^{+}}$}\xspace}
\newcommand{\rmOmegas}  {\mbox{$\mathrm {\Omega^{-} \kern-0.3em +  \kern-0.1em \overline{\Omega}^{+}}$}\xspace}
\newcommand{\rmOmegaPM} {\mbox{$\mathrm {\Omega^{\pm}}$}\xspace} 
% \newcommand{\rmOmegaPM} {\mbox{$\kern+0.0em \mathrm{\vcenter{\offinterlineskip \vskip-1.0ex\hbox{\tiny \kern-0.05em- \kern-0.3em- \kern-0.3em- \kern-0.3em-} \vskip+0.1ex\hbox{$\Omega$}}^{\protect \tiny ^\fivedots \kern-0.8em _{\relbar} \kern-.3em}}$} }
% \newcommand{\rmOmegaPM} {\mbox{$\kern-0.1em \mathrm{ \protect \overset{_\hbox{\Cutline}}{\Omega}^{\protect \underline{\fivedots}} \kern-0.1em}$}}


\newcommand{\rmDeuton}   {\mbox{$\mathrm {d}$}}
\newcommand{\rmTriton}   {\mbox{$\mathrm {t}$}}
\newcommand{\rmHeThree}  {\mbox{$\mathrm {^3He}$}}
\newcommand{\rmHeFour}   {\mbox{$\mathrm {^4He}$}}

\newcommand{\rmJpsi}    {\mbox{$\mathrm{J\kern-0.05em /\kern-0.05em\psi}$}}
\newcommand{\rmPsiTwoS} {\mbox{$\mathrm {\psi(2S)}$}}
\newcommand{\rmChicZero}{\mbox{$\mathrm {\chi_{c_0}}$}}
\newcommand{\rmChicOne} {\mbox{$\mathrm {\chi_{c_1}}$}}
\newcommand{\rmChicTwo} {\mbox{$\mathrm {\chi_{c_2}}$}}
\newcommand{\rmChicJ}   {\mbox{$\mathrm {\chi_{c_J}}$}}

\newcommand{\rmLambdaC} {\mbox{$\mathrm {\Lambda}_{c}^{+}$}}

\newcommand{\rmDzero}   {\mbox{$\mathrm {D}^{0}$}}
\newcommand{\rmDzeroBar}{\mbox{$\mathrm {\overline{D}}^{0}$}}
\newcommand{\rmDplus}   {\mbox{$\mathrm {D}^{+}$}}
\newcommand{\rmDminus}  {\mbox{$\mathrm {D}^{+}$}}
\newcommand{\rmDpm}     {\mbox{$\mathrm {D}^{\pm}$}}
\newcommand{\rmDstar}   {\mbox{$\mathrm{D}^*\mathrm{(2010)}^+$}}
\newcommand{\rmDs}      {\mbox{$\mathrm {D}^{+}_{s}$}}

\newcommand{\rmBzero}       {\mbox{$\mathrm {B^{0}}$}}
\newcommand{\rmBplus}       {\mbox{$\mathrm {B^{+}}$}}
\newcommand{\rmBminus}      {\mbox{$\mathrm {B^{-}}$}}
\newcommand{\rmBplusMinus}  {\mbox{$\mathrm {B^{\pm}}$}}
\newcommand{\rmBzeroS}      {\mbox{$\mathrm {B^{0}_s}$}}


\newcommand{\rmUpsOneS}         {\mbox{$\mathrm {\Upsilon(1S)}$}}
\newcommand{\rmUpsTwoS}         {\mbox{$\mathrm {\Upsilon(2S)}$}}
\newcommand{\rmUpsThreeS}       {\mbox{$\mathrm {\Upsilon(3S)}$}}
\newcommand{\rmUpsTwoThreeS}    {\mbox{$\mathrm {\Upsilon(2S,3S)}$}}
\newcommand{\rmUpsnS}           {\mbox{$\mathrm {\Upsilon(nS)}$}}
\newcommand{\rmUpsOneTwoThreeS} {\mbox{$\mathrm {\Upsilon(1S,2S,3S)}$}}

\newcommand{\rmPhoton}      {\mbox{$\mathrm {\gamma}$}}
\newcommand{\rmWplus}       {\mbox{$\mathrm {W^{+}}$}}
\newcommand{\rmWminus}      {\mbox{$\mathrm {W^{-}}$}}
\newcommand{\rmWPlusMinus}  {\mbox{$\mathrm {W^{\pm}}$}}
\newcommand{\rmZzero}       {\mbox{$\mathrm {Z}$}}



\newcommand{\qqbar}             {\mbox{$q\overline{q}$}}
\newcommand{\ccbar}             {\mbox{$c\overline{c}$}}
\newcommand{\bbbar}             {\mbox{$b\overline{b}$}}
\newcommand{\DDbar}             {\mbox{$\mathrm {D\overline{D}}$}}




% Numbering of lines
    \modulolinenumbers[2]
    % \pagewiselinenumbers
    \switchlinenumbers   % allow to put line numbers on the outer margins
    \linenumbers


% Draft ID
    \def\currentVersion{\version{$\alpha.1$}{\input{|"git log -n1 | awk '/commit/ {print $2}' | cut -c 1-7"}}}
%     \def\currentVersion{\version{$\beta.0$}{dummy}}
%     \def\currentVersion{\version{$\beta.0$}{\input{|"svn info | awk -F : '/vision/ {print $2}' | head -n 1"}}}
        % Trick to get the git rev:
        %   do a grep (awk in fact) through a bash command called (even, piped) from LaTeX
        % Inconvenience:
        %   One needs to authorise the bash command to be executed, with --shell-escape
        %   = pdflatex --shell-escape aliceCDSpreprint_hV0Correl_PbPb_Master.tex
    \backgroundsetup{contents=\currentVersion}
        % define the draft watermark from the background package

    

    
        % Trick to get the svn rev :
        %   do a grep (awk in fact) through a bash command called (even, piped) from LaTeX 
        % Inconvenience :
        %   One needs to authorise the bash command to be executed, with --shell-escape
        %   = pdflatex --shell-escape aliceCDSpreprint_Master.tex
    \backgroundsetup{contents=\currentVersion}
        % define the draft watermark from the background package

\begin{document}%

\pagestyle{MainStyle}

%______________________________________________________________________________
%______________________________________________________________ Title


\begin{titlepage}
    %
    \PHyear{2021}
    \PHnumber{XXX}      % required, will be obtained from PH
    \PHdate{\monthname[\month]}  % required, will be obtained from PH
    %

    %%% Put your own title + short title here:
    \title{Study of correlated production of strangeness via the measurement of the yield ratio of \rmOmegaPM\ and \rmPhiMes\ with \pp\ data of LHC run II}
    \ShortTitle{Short title}   % appears on right page headers

    %%% Do not change the next lines
    %\Collaboration{ALICE Collaboration\thanks{See Appendix~\ref{app:collab} for the list of collaboration members}}
    \ShortAuthor{ALICE Collaboration} % appears on left page headers, do not change

    

    \Collaboration{%    
    \thanks{by alphabetical order}
        Antonin Maire$^{1,2}$, Boris Hippolyte$^{1.2}$, Romain Schotter$^1$\\        
            \footnotesize \color{Gray}
            1. Université de Strasbourg, IPHC, 23 rue du Loess 67037 Strasbourg, France. \\
            2. CNRS, UMR7178, 67037 Strasbourg, France.\\
            ~\newline
            \normalsize \color{Black}    
            \vspace*{0.2cm}~
          {\Large ALICE analysis note}\\
          \vspace{0.5cm}~
          PWG-LF / PAG-Strangeness
    }
    
    
    \begin{abstract}

        \begin{center}
            \textcolor{Gray}{\currentVersion}
        \end{center}
        
        \vspace*{0.2 cm}


        \vspace*{0.5 cm}

        \textit{Keywords:}\\        
        \null~~~~ \rmAomegaP, \rmOmegaM, \rmPhiMes,\\ 
        \null~~~~ALICE, \pp, \sqrtS\ = 13 \tev\ (2016+2017+2018), LHC, runII, strangeness.


    \end{abstract}

\end{titlepage}
\setcounter{page}{2}


%______________________________________________________________________________
%______________________________________________________________ Core document


\hypertarget{LinkToTOC}
\tableofcontents
\cleardoublepage 
\phantomsection 


\part{Brainstorming and considerations (for PC internal use)}

%//------ Section 00 -------------------------------------------------------------------------------------------------
\section{Brainstorming}
\label{sec:Section00}
%//-----------------------------------------------------------------------//

\subsection{List of various strategical ideas}
\label{sec:Section00.a-} % 

\newpage
\part{Data sets and analysis strategy}

\newpage


%//------ Section 01 -------------------------------------------------------------------------------------------------
\section{Introduction}
\label{sec:Section01-Intro}
%//-----------------------------------------------------------------------//

Colliding heavy nuclei at extremely high energy offers the possibility to form a very hot, dense and strongly interacting state of matter, called Quark Gluon Plasma (QGP). This exists for about $10^{-23}$ s \ref{bjorken_highly_1983}, which is far shorter than the readout time of the fastest detector -- the observation of such a state is therefore based on its remnants, meaning the collision products. QGP existence is associated to several signatures ; most of them have also been observed in high multiplicity proton-proton (pp) collisions, where no QGP is foreseen \ref{cms_collaboration_evidence_2015}\ref{alice_collaboration_enhanced_2017}\ref{alice_collaboration_multiplicity_2016}. This questions thorougly our representation of this state of nuclear matter, and calls for a review of the concepts : either the AA collisions/QGP physics picture must be further rooted on pp collisions, or pp QCD physics must be enriched with AA considerations. One way or the other, a better description of the continuum between pp and AA collisions is needed. Nowadays, several phenomenological models are following this approach, but more experimental input are needed (Ref needed). More precisely, this input takes the shape of multi-differential measurements of identified hadrons as a function of the rapidity, the transverse momentum and the multiplicity in charged particles. This type of analysis relies on excellent particle identification capabilites, which are provided by the ALICE detector.

One of the signature usually associated to QGP is the increase of the production of strange quarks, also called strangeness enhancement. Since strangeness is very sensitive to collectivity and the enhancement of the latter is the steepest for pp collisions, this signature might be the way to shed light on the collision dynamic in small systems. Thus, a focus will be given on measurement of the correlated production of strange hadrons in pp collisions. 

An example of a such measurement is proposed by Christian Bierlich : thanks to the recent developments in \Pythiaeight -- namely the color rope and color shoving mechanisms -- it is predicted that i) the \rmOmegaM [$sss$] abundancy increases when an \rmPhiMes [$s \bar{s}$] is present within the event, and ii) this increase is more prominent as the gap in rapidity between the two particles reduces. 

In this analysis, we propose to study the correlated production of strangeness via the measurement of the yield ratio of (\rmOmegas) and \rmPhiMes in proton-proton collisions at a center-of-mass energy $\sqrt{s} = 13 $ TeV from the LHC run II, using the ALICE detector. A comparison with various Monte-Carlo (MC) generators comes to complement the results presented in this paper. For the sake of brievty, ( \rmOmegas) will be denoted \rmOmegaPM, unless otherwise stated. This paper is organized as follows. In section \ref{sec:Section02}, the dataset used in this analysis is presented, followed by the event selections in section \ref{sec:Section03}. Further details about the analysis are showed in section \ref{sec:Section04} with the track and topological selections, and in section \ref{sec:Section05}, systematic study is exposed. Finally, the results are given in section \ref{sec:Section06}, accompanied by the comparison with phenomenological models and the discussion in section \ref{sec:Section07}.



 % Introduction
~\newpage
%//------ Section 02 -------------------------------------------------------------------------------------------------
\section{Data sets in \pp}
\label{sec:Section02}
%//-----------------------------------------------------------------------//

\subsection{SbSection}
\label{sec:Section02.a-}

\subsubsection{Sub-Sbsection} % Data sets
%//------ Section 02 -------------------------------------------------------------------------------------------------
\section{Event selections}
\label{sec:Section03}
%//-----------------------------------------------------------------------//

Event selection was based on a minimum bias trigger (kINT7), requiring a hit in both V0-A and V0-C counters in coincidence with the arrival of proton bunches in the two directions, which leads to the rejection of beam-gas background events. Such background events are then further removed offline, using timing information provided by the V0, and correlation between clusters and reconstructed tracklets (\ref{alice_collaboration_performance_2014}). A constrain on the longitudinal position of the reconstructed vertex ($|z|$ < 10 cm, with $z$ the position along the beam axis) is applied in order to reduce the number of unwanted collisions and ensure that reconstructed tracks are contained inside the acceptance of ALICE barrel. Only events having at least one \rmOmegaPM candidate are selected ; the set of selections for these candidates, as well as the reason for such a request, are presented in section \ref{sec:Section04}.

In order to study multiplicity dependence, the event sample is divided into multiplicity classes, based on cuts on the total charged deposited in the V0 detector (V0M amplitude) or on the number of tracklets in the region \abspseudorap < 1 (\Ntracklet) ; the V0M amplitude varies proportionnally with the number of charged particle produced in the pseudo-rapidity range of the V0 detector (Ref V0 Performance). These classes are reported in \tab \ref{MultClass}, with their corresponding percentages of the total cross section \INELZero, \sigmaINELZero.

\begin{table}[h]
    \centering
    \begin{tabular}{c|ccccc}
    \noalign{\smallskip}\hline \hline \noalign{\smallskip}
    Multiplicity Class & I & II & III & IV & V \\
	\sigmaINELZero & 0-0.95\% & 0.95-4.7\% & 4.7-9.5\% & 9.5-14\% & 14-19\% \\	        	\noalign{\smallskip}\hline \hline \noalign{\smallskip}
	Multiplicity Class & VI & VII & VIII & IX & X \\
	\sigmaINELZero & 19-28\% & 28-38\% & 38-48\% & 48-68\% & 68-100\% \\
    \noalign{\smallskip}\hline \hline \noalign{\smallskip}
    \end{tabular}
    \caption{Event multiplicity classes used in this analysis and their corresponding \sigmaINELZero. (\ref{alice_collaboration_production_2020})}\label{MultClass}
\end{table}
 % Event selections
%//------ Section 04 -------------------------------------------------------------------------------------------------
\section{Track selections and topological selections}
\label{sec:Section04}
%//-----------------------------------------------------------------------//

Measurement of primary particles are presented in this work. A primary particle corresponds to any hadron whose mean proper lifetime is larger than 1 cm, produced either directly during the collision or by the decay of particles with lifetime \cTau < 1 \cm originating from the interaction point \ref{noauthor_alice_2017}. Because of the short lifetime of the particle of interest -- namely \rmOmega and \rmPhiMes --, they must be reconstructed based on their decay products. Both are reconstructed at mid-rapidity (\absrap < 0.5), via an invariant mass analysis.

\subsection{\rmOmega case}
\label{sec:Section04.a-}

\subsubsection{Candidate selections}

The multi-strange hadrons \rmOmegaM and \rmAomegaP are studied in the following decay channel :

\rmOmegaM [$sss$] $\rightarrow$ \rmLambda [$u d s$] \Kminus [$\bar{d} s$]  \qquad \textsc{B.R. 67.8 \%}\\
\rmAomegaP [$\bar{s}\bar{s}\bar{s}$] $\rightarrow$ \rmAlambda [$\bar{u}\bar{d}\bar{s}$] \Kplus [$u\bar{s}$] \qquad \textsc{B.R. 67.8 \%} \\
\\
\rmLambda [$u d s$] $\rightarrow$ \proton [$uud$] \piMinus [$\bar{u} d$] \qquad \textsc{B.R. 63.9 \%}\\
\rmAlambda [$\bar{u}\bar{d}\bar{s}$] $\rightarrow$ \pbar [$\bar{u} \bar{u} \bar{d}$] \piPlus [$u \bar{d}$] \qquad \textsc{B.R. 63.9 \%}

The \rmLambda and \rmOmega being hyperons, they follow a V-shaped decay topology, typical of this family of particles. The multi-strange hadron decay into a kaon and a \rmLambda. The latter being electrically neutral, only the charged meson is detected at this stage ; the meson is considered as a bachelor particle. Further away, the baryon daughter decays into two oppositely charged particles ($V^0$ decay) : a proton and a pion. Depending on their electric signs, one is called positive and the other negative. Thus, the \rmOmega undergoes two step decay process, known as cascade decay. In the following, the usage of the term \textit{cascade} can refer to \rmOmega, and similarly the term \textit{V0} to \rmLambda.

Cascade reconstruction is done by associating three tracks ; these tracks must first pass a set of selections. Only the associations matching the topological selections are considered. The selections used in this work are summarised in \tab \ref{tab:OmegaSel}. 

\begin{table}[h]
    \centering
    \begin{tabular}{c|c}
    \noalign{\smallskip}\hline \hline \noalign{\smallskip}
    \bf Topological variable & Selections \rmOmegaM (\rmAomegaP) \\
    \noalign{\smallskip}\hline \hline \noalign{\smallskip}
    
    \multicolumn{2}{l}{\textbf{V0}} \\
    V0 radius (cm) & > 1.1\\
    V0 CosPA & > 0.97\\
    |$m$($V0$) - \mPDG\rmLambda| (\gmass) & < 0.008 \\
    DCA pos. to prim. vtx (cm) & > 0.03(0.04) \\
    DCA neg. to prim. vtx (cm) & > 0.04(0.03) \\
    DCA V0 to prim. vtx (cm) & > 0.06 \\
    DCA between V0 daughters (std dev) & < 1.5 \\
    \noalign{\smallskip}\hline \noalign{\smallskip}
    
    \multicolumn{2}{l}{\textbf{Cascade}} \\
    Casc. radius (cm) & > 0.5 \\
    Casc. Lifetime (cm) & <  3 $\times$ 2.461 \\
    DCA bach. to prim. vtx (cm) & > 0.04 \\
    DCA between the casc. daughters (std dev) & < 1.3 \\
    Casc. CosPA & > 0.999 \\
    Bach-baryon PA & > 0.04 \\
    
    \noalign{\smallskip}\hline \hline \noalign{\smallskip}
    \bf Track variable & Selections \rmOmegaM (\rmAomegaP) \\
    \noalign{\smallskip}\hline \hline \noalign{\smallskip}
    Daughter pseudo-rapidity interval & \abspseudorap < 0.8 \\
    TPC refit & \CheckGr \\
    Kink Topology & \NoWay \\
    Nbr of crossed rows & > 70 \\
    TPC $dE/dx$ & < 3 $\sigma$ \\

    \noalign{\smallskip}\hline \hline \noalign{\smallskip}
    \bf Candidate variable & Selections \rmOmegaM (\rmAomegaP) \\
    \noalign{\smallskip}\hline \hline \noalign{\smallskip}    
    Cascade \pT interval (\gmom) & 1 < \pT < 5 \\
    Cascade rapidity interval & \absrap < 0.5 \\
    |$m$(hyp. \rmXiPM) - \mPDG\rmXi| (\gmass) & > 0.008 \\
    MC association (MC only) & Correct identity assumption on casc. and daughters \\ 
    \noalign{\smallskip}\hline \hline \noalign{\smallskip}
    \end{tabular}
    \caption{Summary of the topological selections and track selections used for the reconstruction of \rmOmegaPM.}\label{tab:OmegaSel}
\end{table}

Only tracks with at least 70 crossed rows in the TPC out of 159 are retained in the analysis. Each decay products must be contained inside the fiducial tracking volume, \abspseudorap < 0.8. All the tracks must have been refitted, with the Kalman filter in the TPC, backwards to the primary vertex. The energy loss ($dE/dx$) of all decay products is requested to be compatible in $\pm$ 3 $\sigma$ with the considered PID hypothesis. Tracks exhibiting a kink topology are discarded.

Cascade candidates are required to be in the rapidity window \absrap < 0.5. \rmOmega candidates whose reconstructed mass under \rmXi hypothesis lies within a window of $\pm$ 8 \mmass around the \rmXi PDG mass are rejected. A similar selection is also applied on the V0 ; only V0 candidates compatible with the \rmLambda PDG mass within $\pm$ 8 \mmass are selected. A set of topological selections are used in order to identify V0 and cascade decay topologies. 

\subsubsection{Raw signal extraction}

The signal extraction is performed as a function of \pT. A fit is performed using a modified gaussian for the signal and a first order polynomial for the background (Ref formula modified gaussian). This allows to extract the mean value ($\mu$) and the width of the gaussian($\sigma$). Signal region is defined in $\mu \pm 5 \sigma$ ; background region corresponds to bands, of the same width, surrounding on both side the signal area, that is $]\mu -10 \sigma ; \mu -5 \sigma ] \cup [ \mu + 5 \sigma ; \mu + 10 \sigma[$. The raw signal is then extracted from the integral of the modified gaussian within the signal region. 

\begin{equation}
\frac{\text{d}N}{\text{d}m_{\Lambda K}} = A \cdot \text{exp}[-0.5 \cdot x^{(1 + \frac{1}{1+0.5 \cdot x})}], \ x = \left | \frac{m_{\Lambda K} - \mu}{\sigma} \right |
\end{equation}\label{eq:OmegaSignal}

An example of an invariant mass distribution can be found in \fig. A peak of over-population emerges around the PDG mass of the \rmOmega, and one may notice that purity exceeds the 90\%. Such a pure sample has been obtained by restricting the range of possible value on the cosine of pointing angle of the cascade daughters ; this tight selection value corresponds to the one displayed in \tab \ref{tab:OmegaSel}.

\subsection{\rmPhiMes case}
\label{sec:Section04.b-}

\subsubsection{Candidate selections}

\rmPhiMes resonances are studied in the following decay channel :

\rmPhiMes [$s\bar{s}$] $\rightarrow$ \Kplus [$u\bar{s}$] \Kminus [$\bar{u} s$]  \qquad \textsc{B.R. 49.2 \%}

This resonance is reconstructed by forming pairs of tracks of opposite charge ; in the following, these pairs will be called "same-event pairs" . Due its short lifetime of 46.4 \fm \ref{noauthor_pdglive_nodate}, misassociations can not be discarded with a set of geometrical selections like in \rmOmega reconstruction, leading a large combinatorial background. This background is evaluated and then substracted using two techniques presented later in this section.

Tracks are being selected using standard ITS/TPC track cuts from 2011. High-quality tracks are ensured by restricting the number of crossed rows in TPC to be greater 70 (out of 159 pad rows) ; its ratio to the number of findable clusters should exceed 0.8. This is also achieved via cuts on $\chi^2$ per cluster. The latter  quantity on reconstructed tracks in the TPC and in the ITS are requiered to be smaller than 4 and 36 respectively ; $\chi^2$ value between TPC track constrained to the SPD vertex and global track must be inferior to 36. Only tracks following this requierements are retained : at least one hit in the innermost layer of the ITS, and the distance of closest approach to the primary vertex (DCA) lower than $0.0105 + 0.035 \pT^{-1.01}$ cm in the transverse plane and 2 cm along the longitudinal direction. In addition, kink decay topologies are discarded. This allows to reject secondary particles coming from weak decay. Each track has to lie in the pseudo-rapidity window \abspseudorap < 0.5 and to carry a transverse momentum between 0.15 and 4 \gmom. Energy loss in the TPC must be consistent with the PID hypothesis within $\pm$ 2 $\sigma$. If track matches a hit in the TOF, time of flight measurement is then used to select tracks compatible in $\pm$ 3 $\sigma$ with particle's species hypothesis.
Only resonance candidates sitting in the rapidity interval \absrap < 0.5 are selected. 

\begin{table}[h]
    \centering
    \begin{tabular}{c|c}
    \noalign{\smallskip}\hline \hline \noalign{\smallskip}
    \bf Track variable & Selections \rmPhiMes \\
    \noalign{\smallskip}\hline \hline \noalign{\smallskip}
    \pT interval (\gmom) & 0.15 < \pT < 4 \\
    Daughter pseudo-rapidity interval & \abspseudorap < 0.8 \\
    TPC $\text{d}E/\text{d}x$ & < 2 $\sigma$ \\    
    TOF $\beta$ (veto only) & < 3 $\sigma$ \\    
    \noalign{\smallskip}\hline \noalign{\smallskip}
    
    \multicolumn{2}{l}{\textbf{Standard ITS/TPC Cuts 2011}} \\
    ITS refit & \CheckGr \\
    TPC refit & \CheckGr \\
    Kink Topology & \NoWay \\
    Nbr of crossed rows & > 70 \\
    Nbr of crossed rows over findable clusters & $\geq$ 0.8 \\
	$\chi_\textsc{TPC}^2$ & < 4 \\
	$\chi_\textsc{TPC-CG}^2$ & < 36 \\
	$\chi_\textsc{ITS}^2$ & < 36 \\
	Nbr of clusters in SPD & $\geq$ 1 \\
	\textsc{DCA}xy (cm) & < 0.0105 + 0.035 \pT$^{-1.01}$ \\
	\textsc{DCA}z (cm) & < 2 \\
	DCAToVertex2D & \NoWay \\
	RequireSigmaToVertex & \NoWay \\
    \noalign{\smallskip}\hline \hline \noalign{\smallskip}
    \bf Candidate variable & Selections \rmPhiMes \\
    \noalign{\smallskip}\hline \hline \noalign{\smallskip}    
    Resonance rapidity interval & \absrap < 0.5 \\
    MC association (MC only) & Correct identity assumption on casc. and daughters \\ 
    \noalign{\smallskip}\hline \hline \noalign{\smallskip}
    \end{tabular}
    \caption{Summary of the track and candidate selections used for the reconstruction of \rmPhiMes.}\label{tab:PhiSel}
\end{table}

To tackle the combinatorial background, two approaches have been followed in this work. In the "mixed-events" method, each track from one event is paired with tracks of opposite charge coming from five different events in order to build uncorrelated pairs. Only the events whose difference in longitudinal position of primary vertex remains in $\pm$ 1 cm and multiplicity percentile, calculated from the V0M amplitude, coincides with $\pm$ 10\% are considered for the mixing. The mixed-event distribution is normalized by two times the number of mixed-events. In the second approach, one track from the resonance candidate (either the positive or the negative daughter) is rotated by 180 degree along the $z$ axis. Doing so, correlation between the two daughters is broken, leading to an uncorrelated pair. In this case, normalization factor is calculated such that the integral of the same-event and rotated-event distribution in the region 1.04 < $m_{KK}$ < 1.06 \gmass coincides. 

After normalization, the final invariant mass distribution is obtained by substraction of the same-event pair distribution by the uncorrelated pair distribution. In the following, the remaining background will be named \textit{residual background}.

\subsubsection{Raw signal extraction}

For each \pT bin, in the mass window 0.990 < $m_{KK}$ < 1.060 \gmass, signal peak is fitted by a Voigt function (convolution of Breit-Wigner, for the ideal signal, and a Gaussian, for detector resolution) added by a first order polynomial for the residual background. Raw signal is then extracted by the integral of the Voigt function within the fitting range.

\begin{equation}
\frac{\text{d}N}{\text{d}m_{KK}} = \frac{A \Gamma}{(2\pi)^{3/2} \sigma} \int\limits_{-\infty}^{\infty} \text{exp} \big [ -\frac{(m_{KK} - m')^2}{2\sigma^2} \big ] \frac{1}{(m' - \mu)^2 + \Gamma^2/4} \text{d} m'
\end{equation}\label{eq:PhiSignal}

Width of the \rmPhiMes is fixed to its nominal value $\Gamma$ = 4.29 \mev , and the resolution $\sigma$ is kept as a free parameter. \fig illustrates an example of the invariant mass distribution obtained with the aforementionned procedure, on which a \rmPhiMes peak arises on top of a residual background. 

\subsection{Yield extraction}
\label{sec:Section04.c-}

For each \pT bin, raw yield is calculated from the previously extracted raw signal using the expression \eq \ref{eq:RawYield} :
\begin{equation}
\frac{\text{d}^2 N}{\text{d}\pT \text{d} y} \bigg\rvert_\textsc{raw} = \frac{S_\text{raw}}{\Delta \pT \Delta y}
\end{equation}\label{eq:RawYield}

Raw particle yields are then corrected in acceptance (A) and efficiency ($\epsilon_\textsc{rec}$) using a MC data sample : the enriched one for the \rmOmega, and the general purpose one for the \rmPhiMes. The goal is to estimate the efficiency factor, $\epsilon$, in \eq \ref{eq:Efficiency}. The numerator corresponds to the number of reconstructed \rmOmega/\rmPhiMes; reconstruction is performed as if it would be real data, with an additionnal request : only reconstructed cascade/resonance associated to a generated \rmOmega/rmPhiMes are selected. This ensures the efficiency factor to be defined between 0 and 1. Denominator contains the number of generated particles in the selected events, coming directly from MC truth. To avoid bin flow effect, the same set of selection must be applied on both numerator and denominator ; this concerns the cuts on the rapidity and on the transverse momentum. \fig ? and ? show efficiency factors as a function of \pT, for \rmOmega and \rmPhiMes respectively.

\begin{equation}
\epsilon = A\times \epsilon_\textsc{rec} \times \text{B.R.} = \frac{\text{Number of reconstructed \hPM}}{\text{Number of generated \hPM}}
\end{equation}
\label{eq:Efficiency}

Once efficiency factors are evaluated for each \pT bin, corrected yields is given by dN/dptdy$_\textsc{raw}/\epsilon$. Uncertainties on $\epsilon$ are not propagated, they will be included in the systematic uncertainties. Once the yield is extracted for each \pT bin, they are all summed and uncertainties are added in quadrature. 

Following the prediction to be tested, only events presenting at least one \rmPhiMes resonance should be selected, and \rmOmega yield should be extracted afterwards. As a matter of fact, it is completely equivalent to do the opposite, meaning : filter events with at least one \rmOmega, and then extract \rmPhiMes. However, it has been shown, in sections \ref{sec:Section04.a-} and \ref{sec:Section04.b-}, that both species have distinct purities. Consequently, even if the two approaches of the \Pythiaeight prediction are equivalent, one may introduce more noise than the other. With a purity exceeding 90\%, the second approach appears to be a relevant choice.  % Track and topological selections

%//------ Section 05 -------------------------------------------------------------------------------------------------
\section{Systematic studies}
\label{sec:Section05}
%//-----------------------------------------------------------------------//

\subsection{SbSection}
\label{sec:Section05.a-}

\begin{table}[h]
    \centering
    \scalebox{0.8}{
    \begin{tabular}{|c|c|c|c|c|}
    \hline
         Systematic type &  very loose & loose & tight & very tight\\
         \hline
         \hline
         V0 cos(PA) & $>$ 0.95 & $>$ 0.96 & $>$ 0.98 &$>$ 0.985 \\
         Cascade cos(PA) & $>$ 0.95 & $>$ 0.96 & $>$ 0.98 &$>$ 0.99 \\
         DCA bachelor-V0(cm) & $<$ 2.0 & $<$ 1.8 & $<$ 1.0 &$<$ 0.6 \\
         DCA V0 daughters ($\sigma$) & $<$ 2.0 & $<$ 1.8 & $<$ 1.3 &$<$ 1.0 \\
         Cascades' transverse radius(cm) & $>$ 0.3 & $>$ 0.4 & $>$ 0.6 &$>$ 0.7 \\
         V0's transverse radius(cm) & $>$ 0 & $>$ 1.0 & $>$ 2.5 &$>$ 6 \\
         DCA bachelor-PV(cm) & $>$ 0 & $>$ 0.03 & $>$ 0.07 &$>$ 0.10 \\
         DCA V0-PV(cm) & $>$ 0 & $>$ 0.05 & $>$ 0.08 &$>$ 0.10 \\
         DCA V0's meson daughter -PV(cm) & $>$ 0.02 & $>$ 0.03 & $>$ 0.10 &$>$ 0.30 \\
         DCA V0's baryon daughter -PV(cm) & $>$ 0.0 & $>$ 0.02 & $>$ 0.05 &$>$ 0.10 \\
         \hline
         V0 mass window (MeV/c$^2$) & $\pm$ 0.007 & $\pm$ 0.0065 & $\pm$ 0.006 &$\pm$ 0.005 \\
         TPC PID ($\sigma$ from BB curve) & $\pm$ 3. & $\pm$ 2.5 & $\pm$ 2.0 &$\pm$ 1.5 \\
         \hline
    \end{tabular}}
    \caption{Topological cuts}
    \label{tab:TopologicalVariation}
\end{table}

\subsubsection{Sub-Sbsection} % Systematic checks

\newpage
\part{Results and discussion}

\newpage
%//------ Section 06 -------------------------------------------------------------------------------------------------
\section{Results}
\label{sec:Section06}
%//-----------------------------------------------------------------------//

\subsection{SbSection}
\label{sec:Section0x.a-}

\subsubsection{Sub-Sbsection} % Results
\newpage
%//------ Section 07 -------------------------------------------------------------------------------------------------
\section{Discussion}
\label{sec:Section07}
%//-----------------------------------------------------------------------//

\subsection{SbSection}
\label{sec:Section0x.a-}

\subsubsection{Sub-Sbsection} % Discussion


% http://en.wikibooks.org/wiki/LaTeX/Bibliography_Management#Bibliography_in_the_table_of_contents
\cleardoublepage % pour fermer le dernier paragraphe du chapitre précédent (ccl°)
\phantomsection % With hyperref package, you should also use \phantomsection command to enable hyperlinking from the table of contents to bibliography.





%______________________________________________________________________________
%______________________________________________________________ Acknowledgements

\newenvironment{acknowledgement}{\relax}{\relax}
\addcontentsline{toc}{section}{Acknowledgements}
\begin{acknowledgement}
\section*{Acknowledgements}


For Romain Schotter, this work of the Interdisciplinary Thematic Institute QMat, as part of the ITI 2021 2028
program of the University of Strasbourg, CNRS and Inserm, has been supported by IdEx Unistra
(ANR 10 IDEX 0002), and by SFRI STRAT’US project (ANR 20 SFRI 0012) and EUR
QMAT ANR-17-EURE-0024 under the framework of the French Investments for the Future
Program.
    
%% complemented here with the QMat aknowledgements
%%%%%%% done by webmaster team, a priori
\end{acknowledgement}

%______________________________________________________________________________
%______________________________________________________________ Bibliography
% (In case of using bibtex generate the bbl requested by arXiv)
\footnotesize

%\bibliographystyle{utphys}   
\bibliographystyle{myJHEP} % http://jhep.sissa.it/jhep/help/JHEP_TeXclass.jsp


% http://en.wikibooks.org/wiki/LaTeX/Bibliography_Management#Bibliography_in_the_table_of_contents
\cleardoublepage % pour fermer le dernier paragraphe du chapitre précédent (ccl°)
\phantomsection % With hyperref package, you should also use \phantomsection command to enable hyperlinking from the table of contents to bibliography.

\addcontentsline{toc}{section}{References} 
\bibliography{PrePrint_References.bib}
        
\normalsize


%______________________________________________________________________________
%______________________________________________________________ Appendices


%%%%%%%%% appendix with author list
\newpage
\appendix
%
%\input{}               %%%%%%%%%%% put your appendices here
%

\section*{The ALICE Collaboration}

\addcontentsline{toc}{section}{The ALICE collaboration} 
\label{app:collab}

%\input{authorlist-preprint.tex}  %%%%%%% done by webmaster team

\end{document}
